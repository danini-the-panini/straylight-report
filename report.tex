\documentclass[a4paper]{article}

\usepackage[english]{babel}
\usepackage[utf8]{inputenc}
\usepackage{amsmath}
\usepackage{graphicx}
\usepackage[colorinlistoftodos]{todonotes}

\title{Optimising the Straylight Raytracer}

\author{Daniel Smith}

\date{\today}

\begin{document}
\maketitle

\section{Solution}
The solution used in this optimisation is the utilisation of an R*-Tree. Generally R-Trees organizes the objects within a scene into a tree of axis-aligned bounding-boxes (AABB), where the children of each AABB are situated completely within the bounds of their parent's AABB, and the leaf nodes represent the bounds of each individual object in the scene. The R*-Tree, a specific type of R-Tree, tries to make these AABBs overlap each other as little as possible, thus minimising the amount of redundant intersection tests between each AABB. The aim is to reduce the number of intersection-tests done by removing all AABBs that do not intersect with each ray involved in the testing. R-Trees allow this to be done efficiently by pruning whole branches of the tree based on higher-level bounding boxes, removing many objects from the test with only a single test of their parent.

\section{Experimental Setup}
Experimentation is done by ray-tracing a wide variety of different scenes and measuring the time taken to complete the rendering of each scene. Each scene is rendered twice: once using the original, unoptimised version of the program, and again using the new, optimised version. The experiment is done one two different testing environments in order to obtain a wider range of results. Testing environment A is a 2009 Apple MacBook Pro with a 2GHz Core 2 Duo, 8GB RAM and a 128GB OCZ Vertex3 SSD, running OS X 10.9.2. Testing environment B is a custom-built desktop with a 3.4GHz Core i5-3570, 8GB RAM and a 256GB Samsung 840 EVO SSD, running Ubuntu 13.10.

\section{Analysis}

\section{Results}

\section{Conclusion}

\end{document}